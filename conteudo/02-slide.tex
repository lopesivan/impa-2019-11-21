\section{Método cognitivo}

\begin{frame}
\frametitle{fluxo de vídeo}

Quando a entrada é um fluxo de vídeo, pode-se explorar a correlação temporal
    para facilitar a Reconstrução 3D, garantindo que a reconstrução seja suave
    e consistente em todos os quadros do vídeo corrente.

Além disso, a entrada pode representar um ou vários objetos 3D.  objetos
    pertencentes a categorias de formas conhecidas ou desconhecidas.
\end{frame}


\begin{frame}
\frametitle{Representacão}

A representação é crucial para a escolha da
arquitetura de rede. Ele também afeta a computação
eficiência e qualidade da reconstrução.

\end{frame}

\begin{frame}
\frametitle{arquitetura de backbone}

Uma variedade de arquiteturas de rede tem sido utilizada para melhorar
    implementar o preditor $f$. Uma arquitetura de backbone, que pode ser
    diferente durante o treinamento e o teste, é composto por um codificador
    $h$ seguido de um decodificador $g$, ou seja, $f=g \circ h$.

\end{frame}

\begin{frame}
\frametitle{arquitetura de backbone}

O codificador mapeia a entrada em uma variável latente $x$, referida como
    um vetor de característica ou um código, usando uma sequência de operações
    de pooling, seguidas de operações totalmente conectadas a supervisão,

\end{frame}

\begin{frame}
\frametitle{Gerador}

O decodificador, também chamado de gerador, decodifica o vetor de
    característica na saída desejada usando camadas totalmente conectadas ou
    uma rede de desconvolução (um seqüência de operações de convolução e
    upsampling, também parâmetros podem ser conhecidos ou desconhecidos )

\end{frame}
